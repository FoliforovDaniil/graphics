\section{\label{sec 2}Методика эксперимента}

В данной работе для формирования изоэнтропы разгрузки металла была смоделирована задача о распаде произвольного разрыва. Для этого сжатый металл расширялся в область с меньшим значением плотности системы. В этих условиях, основываясь на гидродинамическом анализе, ударная волна сжатия распространяется в преграду, а в металл – волна разрежения.

В этой работе для анализа термодинамических свойств веществ на атомарном уровне было выполнено 
молекулярно-динамическое моделирование распада произвольного разрыва между исследуемым металлом и преградой. Все расчеты осуществлялись с использованием открытого программного пакета для решения задач молекулярной динамики LAMMPS. Термостатирование моделируемых систем проводилось с применением алгоритма Нозе–Гувера.

Проведение эксперимента по изоэнтропическому расширению в двухфазную область предусматривает разделения моделирования
на два ключевых этапа: подготовка сжатого образца, состоящего из атомов единственного вещества, и преград различной 
плотности; непосредственное моделирование изоэнтропичсекого расширения. В экспериментальных исследованиях получение 
сжатого образца является важным этапом работы, который на практике осуществляется, зачастую, при помощи ударно-взрывного воздействия. Схожий процесс возможно реализовать и в программном пакете LAMMPS с использованием алгоритма, предложенного Оскаром Герреро-Мирамонтесом. Однако в данной работе рассматривается идеализированная постановка задачи, в рамках которой исследуемый образец уже находится в сжатом состоянии с известными термодинамическими значениями. Описанное упрощение позволяет сократить время проведения серий экспериментов и не усложняет процесс подбора начальных условий.

В качестве межчастичного потенциала взаимодействия в данной работе использовался потенциал \acrfull{lj}. Данный потенциал является наиболее широко используемым потенциалом межмолекулярного взаимодействия в истории моделирования, который достаточно хорошо описывает поведение небольших сферических и неполярных молекул. Он был широко изучен в последние десятилетия и может служить реалистичной моделью для изучения фазовых равновесий, процессов фазового перехода, поведения кластеризации или свойств переноса простых жидкостей.

Наиболее часто встречающееся в литературе выражение для данного потенциала записывается следующим образом:
\begin{equation}
    U_{LJ} = 4\varepsilon\left[\left(\frac{\sigma}{r}\right)^{12} - \left(\frac{\sigma}{r}\right)^{6}\right],
    \label{eq:LJ}
\end{equation}
%
где $r$$~-$ расстояние между двумя взаимодействующими частицами, $\varepsilon$$~-$ энергетическая константа (или глубина потенциальной ямы), а $\sigma$$~-$ расстояние, на котором потенциальная энергия межчастичного взаимодействия равна нулю (часто называемая «размер частицы»). Параметры $\varepsilon$ и $\sigma$, определяемые подгонкой к известным свойствам вещества, соответствуют энергии и длине связи соответственно, измеряются экспериментально и являются его характеристиками.

В качестве вещества, находящегося в сжатом состоянии, был взят металл молибден (Mo). Данный выбор был обусловлен наличием экспериментальных, которые необходимо было проанализировать на атомарном уровне для уточнения процессов, происходящих при его попадании в двухфазную область.
% Необходимо внести параметры потенциала ЛД.  
В качестве преграды использовался аргон (Ar). Атомы данного инертного газа является достаточно лёгкими, что позволяет применять его для моделирования преград различной плотности в широком диапазоне значений, включающих предельно малые плотности.
% Необходимо внести параметры потенциала ЛД. 

Процесс моделирования начинался с выбора значений плотности ($\rho$) и температуры ($T$) на фазовой диаграмме с низкими значениями давлений ($P$). Методика эксперимента позволяет сделать это упрощение, ускоряя процесс расчётов и упрощая подбор параметров ячейки симуляции. 

Сама окно моделирования представляет собой прямоугольник значительно вытянутый по одной из осей в длину, на который наложены периодические граничные условия. При задании начальных параметров образца атомы вещества располагаются в узлах ГЦК-решётки, а в случае симуляции преград их распределение определяется произвольным образом. После чего начинается процесс моделирования в NVE-ансамбле с использование термостата. Результатом которого является приведение вещества в состояние термодинамического равновесия с заданными и известными значениями плотности и температуры.

На следующем этапе моделирования распада произвольного разрыва боковые грани подготовленных образца и преграды приводились в соприкосновение (для ясности будем называть оси соприкосновения $Ox$ и $Oy$). Полученная таким образом прямоугольная ячейка представляет вытянутый вдоль одной оси ($Oz$) прямоугольный параллелепипед. Используемая конфигурация позволяет на протяжении достаточного промежутка времени фиксировать распространение ударной и разгрузочной волн вдоль $Oz$. При этом важно отметить, что такая постановка эксперимента накладывают ограничение на использование периодических граничных условий по $Oz$ для предотвращения воздействия мнимых частиц преграды на образец. В связи с этим, для предотвращения разлёта частиц, по краям ячейки моделирования располагаются отражающие "зеркальные" границы. Компонента вектора скорости $z$ движущегося атома, при соприкосновении с ними, меняется на противоположную, при этом модуль вектора скорости сохраняется. Важно отметить, что переодические граничные условия по осям $Ox$ и $Oy$ сохраняются.

Дальнейший процесс моделирования распада произвольного разрыва проводился в микроканоническом ансамбле $NVE$. Для наблюдения за распространением волн ячейка симуляции делилась на тонкие продольные срезы по оси $Oz$. В каждом из которых рассчитывались усреднённые термодинамические величины через равные промежутки времени.  